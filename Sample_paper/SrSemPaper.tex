\documentclass{sig-alternate}
\usepackage{color}
\usepackage[colorinlistoftodos]{todonotes}

\begin{document}
\conferenceinfo{UMM CSci Senior Seminar Conference, December 2015}{Morris, MN}

\title{Thermal interaction in Spatial Augmented Reality}

\numberofauthors{1}

\author{
% The command \alignauthor (no curly braces needed) should
% precede each author name, affiliation/snail-mail address and
% e-mail address. Additionally, tag each line of
% affiliation/address with \affaddr, and tag the
% e-mail address with \email.
\alignauthor
Justin B. YaDeau\\
	\affaddr{Division of Science and Mathematics}\\
	\affaddr{University of Minnesota, Morris}\\
	\affaddr{Morris, Minnesota, USA 56267}\\
	\email{yadea003@morris.umn.edu}
}

\maketitle
\begin{abstract}
A discussion on the use of 3D projectors and thermal sensors to interact with 3D data visualization, without having to actually touch an electronic device. This leads to a cut down on the amount of electronic devices needed for modern life.
\todo[inline]{Subject to change}
\end{abstract}

\keywords{Thermal interaction, 3D Data Visualization, Augmented Reality, Spatial Augmented reality, SAR, AR}

\section{Introduction}
\label{sec:introduction}

\section{Background}
\label{sec:background} 

\subsection{Virtual Reality}
\label{sec:Virtual Reality}

The first thing that comes to mind when thinking about Virtual Reality, VR, is the Oculist Rift. VR has been around for a while, it is not some futuristic technology that we have yet to explore. Some early examples would be the view master, the stereoscopic toy had those circular inserts that required the user to look into a light source to illuminate the picture. Nintendo had its own type of Oculist Rift called Virtual Boy in the 90's.     

\subsection{Augmented Reality}
\label{sec:Augmented Reality}
Augmented Reality, AR, is easily described at augmented the environment of the real world. It differs from VR in the aspect that it is based more in the physical, or real world, instead of only seeing a fully digital one. An example is Google Translate, using the camera from a phone it translates a foreign word that you point at. On the phones screen it overlays the translation onto the sign, billboard, or menu. 

\todo[inline]{3Ds AR Cards?}

\subsection{Spatial Augmented Reality}
\label{sec:Spatial Augmented Reality}
Spatial Augmented Reality, SAR, is similar to AR. The difference is it focuses more on augmenting reality through projection technology, instead of using conventional monitors or other such devices. A good example would be if you have a sandbox with a topographical map overlaid onto it. You could move the same around and the projector would match the peaks and valleys in the sand with the correct topographic overlay for displaying peaks and valleys.  

\subsection{6DOF}
\label{sec:6DOF}
6DOF, 6 degrees of freedom, are the different ways one can move in three dimensional space . From/back, left/right, up/down, roll, pitch, yaw are all the ways to move.
 
\todo[inline]{expand using roll, yaw, and pitch?}


\subsection{Data Visualization}
\label{sec:Data Visualization}
Data visualization is the representation of data visually. Basic examples are pie charts, scatter-plots, bar charts, and the list goes on. Data visualization can be used in any field. From showing the demographic of a neighborhood to how many car crashes happened over a given month.   

\section{Thermal Interaction}
\label{sec:Thermal Interaction}

\section{3D Data Visualization}
\label{sec:3D Data Visualization}

I will now focus on a paper that delves into the use of SAR as a tool for 3D data visualization. As we know from \textbf{Ref Background} SAR uses projectors to augment what is already there. 

\subsection{Visualizing Data}
\label{sec:Visualizing Data}

Visualizing data is a great way to show information very quickly and efficiently. Pictures tend to be more memorable then words.     

\subsection{Applications}
\label{sec:Applications}

The applications that are talked about are Tabletop and CAVE.

\todo[inline]{add pictures of the tabletop or CAVE}

\subsection{Limitations}
\label{sec:Limitations}

The initial experimentation with their prototype showed a number of limitations with that approach. The first is the lighting the room must be controlled, as details in the gradients of the projected data may be lost with too much ambient lighting.
\todo[inline]{finish cleaning this up}


Lighting of the room, solution brighter projectors. Getting the graphics to only hit the cones.   

\subsection{Conclusion}
\label{sec:Conclusion}

\begin{quote}
We have presented the use of SAR as a tool to enhance the process of 3D visualization. The paper defined our proposed use of SAR as a tool for 3D visualization with three different viewpoints. The first was a purposed set of SAR features to improve a user's ability to view, understand, and manipulate 3D visualization data. The second was an example tabletop SAR prototype to demon-strate a number of the possibilities. Finally a collection of three large scale possible applications of SAR was presented.
\end{quote}


\section{Joining}
\label{sec:Joining}

\section{Alternate Interactions}
\label{sec:Alternate Interactions}

\section{Acknowledgments}
\label{sec:Acknowledgments}

\section{References}
\label{sec:References}


\end{document}